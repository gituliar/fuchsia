\documentclass[12pt,twoside,a4paper]{article}

\usepackage{amssymb,amsmath,color,enumitem,hyperref}
\usepackage[affil-it]{authblk}

\bibliographystyle{alpha}
\setcounter{tocdepth}{2}
\setlist[description]{style=nextline}

\definecolor{fuchsia}{RGB}{140,0,130}
\def\fuchsia{\textcolor{fuchsia}{\texttt{fuchsia}}}
\def\F#1{\mathbf{#1}} % use this to style function names
\def\M#1{\mathbb{#1}} % use this to style matrix names
\def\functionitem#1#2{\item[$\F{#1}(#2)$]}

\input include/layout.tex
\input include/title.tex

\begin{document}

\maketitle
\thispagestyle{empty}

\begin{abstract}
\ldots
\end{abstract}
\newpage

\input include/summary.tex
\newpage

\tableofcontents

\section{Introduction}

\section{Usage from SageMath}

\begin{description}

\functionitem{fuchsify}{\M M, x, seed=0}
Reduce a system defined by matrix $\M M$ and independent variable
$x$ to Fuchsian form. That is, make sure that the Poincare rank
of all singularities of the transformed matrix $\M M'$ are
$0$. Return a pair of values: the transformed matrix $\M M'$
and the transformation $\M T$. If the system is irreducible,
raise $FuchsiaError$.

\functionitem{normalize}{\M M, x, epsilon, seed=0}
Transform a Fuchsian system defined by matrix $\M M$, independent
variable $x$ and infinitesimal parameter $epsilon$ to a normalized
form. That is, make sure that eigenvalues of all matrix residues
of the transformed matrix $\M M'$ lie in the range $[-1/2, 1/2)$
in the limit $epsilon\to0$. Return a pair of values: the
transformed matrix $\M M'$ and the transformation $\M T$. If
such transformation can not be found, raise $FuchsiaError$.

\functionitem{factor\_epsilon}{\M M, x, epsilon, seed=0}
Transform a normalized system defined by matrix $\M M$, independent
variable $x$ and infinitesimal parameter $epsilon$ so that the
transformed matrix $\M M'$ is proportional to $epsilon$. Return a
pair of values: the transformed matrix $\M M'$ and the transformation
$\M T$. If such transformation can not be found, raise $FuchsiaError$.

\functionitem{simplify\_by\_jordanification}{\M M, x}
Try to simplify a system defined by matrix $\M M$ and independent
variable $x$ by constant transformations that transform leading
expansion coefficients of $\M M$ into their Jordan forms. Return a
pair of values: the simplified matrix $\M M'$ and the transformation
$\M T$. If none of the attempted transformations reduce the
complexity of $\M M$ (as measured by $\F{matrix\_complexity}$),
return the original matrix and the identity transformation.

\functionitem{simplify\_by\_factorization}{\M M, x}
Try to simplify a system defined by matrix $\M M$ and independent
variable $x$ by a constant transformation that extracts common
factors found in $\M M$ (if any). Return a pair of values:
the simplified matrix $\M M'$ and the transformation $\M T$.

\functionitem{matrix\_complexity}{\M M}
This function is used as a measure of matrix complexity by
$\F{fuchsify}$ and simplification function. Currently it is
defined as the length of textual representation of matrix $\M M$.

\functionitem{balance}{\M P, x_1, x_2, x}
Return a \textit{balance} transformation between points $x=x_1$
and $x=x_2$ using projector matrix $\M P$.

\functionitem{transform}{\M M, x, \M T}
Transform a system defined by matrix $\M M$ and independent
variable $x$ using transformation matrix $\M T$. Return the
transformed matrix $\M M'$.

\functionitem{balance\_transform}{\M M, \M P, x_1, x_2, x}
Same as $\F{transform}(\M M, x, \F{balance}(\M P, x_1, x_2, x))$,
but implemented more efficiently: since the inverse of
$\F{balance}(\M P, x_1, x_2, x)$ is $\F{balance}(\M P, x_2, x_1, x)$,
this function can avoid a time-consuming matrix inversion
operation that $\F{transform}$ must perform.

\functionitem{singularities}{\M M, x}
Find values of $x$ around which matrix $\M M$ has a singularity in
$x$. Return a dictionary with $\{x_i: p_i\}$ entries, where $p_i$
is the Poincare rank of $\M M$ at $x=x_i$. The set of singular points
can include $+Infinity$, if $\M M$ has a singularity at $x\to\infty$.

\functionitem{matrix\_c0}{\M M, x, x_0, p}
Return the 0-th coefficient of the series expansion of matrix
$\M M$ around $x=x_0$, assuming Poincare rank of $\M M$ at that
point is $p$. If $x_0$ is $+Infinity$, return the coefficient
at the highest power of $x$.

\functionitem{matrix\_c1}{\M M, x, x_0, p}
Return the 1-th coefficient of the series expansion of matrix
$\M M$ around $x=x_0$, assuming Poincare rank of $\M M$ at that
point is $p$. If $x_0$ is $+Infinity$, return the coefficient
at the second-to-highest power of $x$.

\functionitem{matrix\_residue}{\M M, x, x_0}
Return matrix residue of matrix $\M M$ at $x=x_0$, assuming that
Poincare rank of $\M M$ at $x=x_0$ is $0$. Return matrix residue
at infinity if $x=+Infinity$.

\functionitem{export\_matrix}{file, \M M}
Write matrix $\M M$ to a file-like object $file$ using MatrixMarket
array format. For description of this format see \cite{MMarket}.

\functionitem{export\_matrix\_to\_file}{filename, \M M}
Write matrix $\M M$ to a file $filename$ using MatrixMarket
array format.

\functionitem{import\_matrix}{file}
Read a symbolic matrix from a file-like object $file$, assuming
it is formatted using MatrixMarket array format.

\functionitem{import\_matrix\_from\_file}{filename}
Read a symbolic matrix from a file $filename$, assuming it is
formatted using MatrixMarket array format.

\end{description}

\section{Usage from the command line}

\section{Summary}

\section*{Acknowledgment}

\bibliography{bibliography}

\end{document}
