\documentclass[12pt,a4paper]{article}

\usepackage{amssymb,amsmath,color,enumitem,hyperref,xspace}
\usepackage[affil-it]{authblk}

\bibliographystyle{halpha}
%\bibliographystyle{JHEP}

\setcounter{tocdepth}{2}
\setlist[description]{style=nextline}

\definecolor{fuchsia}{RGB}{140,0,130}
\def\F#1{\mathbf{#1}} % use this to style function names
\def\M#1{\mathbb{#1}} % use this to style matrix names
\def\functionitem#1#2{\item[$\F{#1}(#2)$]}

\input include/commands.tex

\input include/layout.tex
\input include/title.tex

\begin{document}

\maketitle
\thispagestyle{empty}

\begin{abstract}
We present a program \fuchsia based on the Lee algorithm \cite{Lee15} which, given a system of first-order linear differential equations with rational function coefficients, constructs an equivalent system in the canonical form and corresponding transformation.
After such a reduction is successfully done the system can be trivially solved, which makes a reduction step crucial to obtain the solution.

In principle, \fuchsia can deal with any regular systems, however it's primary task is to reduce differential equations for master integrals which arise from Feynman diagrams.
It ensures that solutions contain regular singularities only due to the analyticity of S-matrix.

We discuss limitations and possible extensions of the proposed implementation.
\end{abstract}
\newpage

\input include/summary.tex
\newpage

\tableofcontents

\section{Introduction}

{\it Feynman integrals calculus: importance, methods to calculate (Smirnov's book \cite{Smi06}), differential equations method (progress \cite{Henn14}) .}

It has passed more than 50 years (?) since Richard Feynman proposed a diagrammatic approach for calculating perturbative processes in quantum field theories.
Since then Feynman integrals calculus has grown to a separate branch of the mathematical physics with a big community of scientists making research in this exciting field.
With no doubt we can say that none of the recent discoveries in the high-energy particle physics could happen without precise theoretical calculations, which are based in their core on the Feynman integrals calculation techniques. 
It is also clear that such techniques will play a key role for discoveries at the present and future high-energy colliders, hence their development and improvement is a very important task.

Recent progress in the computer industry made possible to automate calculation of Feynman integrals; problems which seemed impossible 10 years ago now are successfully solved with state-of-the-art computer algorithms and techniques.
Among the most popular of them is Feynman parameters~\cite{}, sector decomposition~\cite{}, Mellin-Barns~\cite{}, integration-by-parts~\cite{}, differential equations~\cite{}, and others; for a detailed review of these method see~\cite{Smi06}.

{\it Analyticity of S-matrix and Fuchsian systems of DE. Reduction to Fuchsian form as an independent problem (Moser~\cite{Mos59},Barkatou~\cite{BP09}, \texttt{super\_reduce} in Maple), but mistreating $\infty$.}

{\it Integration-by-parts reduction, canonical form (Henn) \cite{Henn13}, Lee algorithm.}

The paper is organized as follows: in Section \ref{sec:2} we introduce notation and definitions followed by a brief review of the three key algorithms of \cite{Lee15} and implemented in \fuchsia.
In Section \ref{sec:3} we describe how to install and use \fuchsia from different environments, depending on your goal and programming experience.
In Section \ref{sec:4} we discuss limitations and possible extensions of the implemented method.


\section{Overview of the method}
\label{sec:2}

In this section, after a few preliminaries we briefly discuss R.Lee's method for reducing a system of ODEs to canonical form.


\subsection{Notation and definitions}

Let us consider a system of (ODEs)
\begin{equation}
\label{eq:df}
  \frac{\D \bar{f}}{\D x} = \M A(x,\eps)\,\bar{f},
\end{equation}
where $\bar{f}(x,\eps)$ is a vector of $n$ unknown functions (e.g., master integrals).
Elements of the $n \times n$ matrix  $\M A(x,\eps)$ are assumed to be rational functions in $x$ of the form
\begin{equation}
\label{eq:axeps}
  A(x,\eps) = \sum_k \frac{1}{(x-x_k)^{1+p_k}} \sum_{i=0}^\infty A_{ik}(\eps) (x-x_k)^i
\end{equation}
with {\em singular points} at $x = x_k$, where $k$ runs over a final set and $x_k$ is a complex number or $\infty$.
{(\em put comment on $\infty$)}

\paragraph{Equivalent systems.}
We can apply a {\em transformation} $\M T$ to the system \eqref{eq:df} by changing a basis $\bar{f}$ to
\begin{equation}
\label{eq:gtf}
  \bar{g} = \M T(x,\eps)\, \bar{f}.
\end{equation}
That leads to the {\em equivalent system} of ODEs
\begin{equation}
  \frac{\D \bar{g}}{\D x} = \M B(x,\eps)\,\bar{g},
\end{equation}
where
\begin{equation}
\label{eq:bta}
  \M B = \M T^{-1} \left( \M A \M T - \frac{\D \M T}{\D x} \right).
\end{equation}
We require that the transformation \eqref{eq:gtf} is also in the rational form, e.g.,  as $\M A$ in \eqref{eq:axeps}, which guarantees that the equivalent matrix $\M B$ is of the rational form too.

\paragraph{Classification of singularities.}
For any system \eqref{eq:df} we can define a rational number
\begin{equation}
  m_k(\M A) = p_k + \frac{r_k}{n} \ge 0
\end{equation}
as the {\em order} or {\em Poincar\'e rank} of $\M A$ at $x=x_k$, where $r_k = r(A_{0k})$ is the algebraic rank of $A_{0k}(\eps)$, $0 \le r_k \le n$.
If $p_k + r_k/n < 0$ we set $m_k(\M A) = 0$.

Under the transformation \eqref{eq:gtf} the order $m_k(\M A)$ may change, i.e., $m_k(\M A) \ne m_k(\M B)$.
Therefore, we introduce the {\em minimum order} or {\em generalized Poincar\'e rank} of the matrix $\M A$ as
\begin{equation}
\label{eq:muk}
  \mu_k(\M A) = \min m_k(\M B), \text{ for } \forall \; \M T.
\end{equation}
If we can decrease a Poincar\'e rank of the matrix at the point $x=x_k$, or in other words if $m_k(\M A) > \mu_k(\M A)$, we say that the matrix is {\em reducible} at this point.

Considering this definition we say that the point $x=x_k$ (the change of variable $x \to 1/x$ allows to classify the point $x=\infty$ in the same way) of $\M A$ is
\begin{itemize}
  \item a {\em regular point}, for $\mu_k(\M A) = 0$ and $m_k(\M A) = 0$;
  \item an {\em apparent singularity} for $\mu_k(\M A) = 0$ and $m_k(\M A) > 0$;
  \item a {\em regular singularity}, for $0 < \mu_k(\M A) \le 1$;
  \item an {\em irregular singularity}, for $\mu_k(\M A) > 1$.
\end{itemize}
We also say that matrix $\M A$ has a {\em Fuchsian form} when it does not contain irregular singularities at any point $x=x_k$, where $x_k$ is a complex number or $\infty$.


\subsection{Reduction to canonical form}

The goal of the method \cite{Lee15}, briefly described below, is to find a rational matrix $\M T(x,\eps)$ which transforms a system of ordinary differential equations with rational coefficients given by the matrix $\M A(x,\eps)$ to the equivalent system given by the matrix $\M M(x,\eps)$ in the {\em canonical form}~\cite{Henn13}, i.e., $\M M(x,\eps) = \eps\, \M M(x)$.
(Here and everywhere else in the text we assume that $\eps \to 0$, e.g., is a dimensional regularization parameter).
Once such a form is found the initial system $\M A$ can be easily solved as a Laurent series in $\eps$ parameter, which is of course the ultimate goal.

The whole method can be divided into three steps~\cite{Lee15}:
\begin{enumerate}
  \item Given a matrix $\M A(x,\eps)$, find a transformation $\M T$ such that the equivalent matrix $\M B(x,\eps)$, given by eq.~\eqref{eq:bta}, has a Fuchsian form.
  \item Given a matrix $\M A(x,\eps)$ in the Fuchsian form, find a transformation $\M T$ such that the equivalent matrix $\M B(x,\eps)$ is normalized, i.e., the eigenvalues of all its residues have the form $n \eps$, where $n$ is integer.
  \item Given a normalized matrix $\M A(x,\eps)$ in the Fuchsian form, find a transformation $\M T$ such that the equivalent matrix $\M B(x,\eps) = \eps \, \M B(x)$.
\end{enumerate}

{\it Notation and definitions \cite{Mos59}: Poincare rank, equivalent and reducible systems, transformation, Fuchsian form (regular, irregular, and apparent singularities), balance.}


\paragraph{The {\em fuchsify} algorithm.}

At first let us make a small remark.
The problem with eq.~\eqref{eq:muk} is that a criterion to directly calculate $\mu_k(A)$ does not exist.
For that reason we can not say whether the point $x=x_k$ is regular or irregular singularity or whether matrix can be reduced to the Fuchsian form.
(However, for the $n$-th order ODE such a criterion exists, which is a generalization of Fuchs' theorem, see~\cite{Mos59}.)

Let us mention, that in regular singular points solutions of \eqref{eq:df} grow at most like a finite power of $\abs{x}$, for that reason we 

{\it Describe key algorithms \cite{Lee15}: fuchsify, normalize, factorize.}

\paragraph{The {\em normalize} algorithm.}

\paragraph{The {\em factorize} algorithm.}


\section{Using \fuchsia}
\label{sec:3}

\subsection{Installation}

\subsection{Usage from the command line}

An easy way to use \fuchsia is to run \code{fuchsia.py} shell script, which has the following invocation signature:

\code{\$ fuchsia.py <action> <options>}

where
\begin{itemize}
  \item \code{<action>} is one of the algorithms described in the previous section, i.e., \code{fuchsify}, \code{normalize}, \code{factorize}, or auxiliary action \code{transform}, which applies a user-defined transformation to the given matrix.
  \item \code{<options>} are action-dependent options described in the help message printed with the help of \code{fuchsia.py --help} command.
\end{itemize}

In the following we provide a complete help information printed by \code{fuchsia.py --help}:

\ldots


\subsection{Usage from \sage}

\begin{description}

\functionitem{fuchsify}{\M M, x, seed=0}
Reduce a system defined by matrix $\M M$ and independent variable
$x$ to Fuchsian form. That is, make sure that the Poincare rank
of all singularities of the transformed matrix $\M M'$ are
$0$. Return a pair of values: the transformed matrix $\M M'$
and the transformation $\M T$. If the system is irreducible,
raise $FuchsiaError$.

\functionitem{normalize}{\M M, x, epsilon, seed=0}
Transform a Fuchsian system defined by matrix $\M M$, independent
variable $x$ and infinitesimal parameter $epsilon$ to a normalized
form. That is, make sure that eigenvalues of all matrix residues
of the transformed matrix $\M M'$ lie in the range $[-1/2, 1/2)$
in the limit $epsilon\to0$. Return a pair of values: the
transformed matrix $\M M'$ and the transformation $\M T$. If
such transformation can not be found, raise $FuchsiaError$.

\functionitem{factor\_epsilon}{\M M, x, epsilon, seed=0}
Transform a normalized system defined by matrix $\M M$, independent
variable $x$ and infinitesimal parameter $epsilon$ so that the
transformed matrix $\M M'$ is proportional to $epsilon$. Return a
pair of values: the transformed matrix $\M M'$ and the transformation
$\M T$. If such transformation can not be found, raise $FuchsiaError$.

\functionitem{simplify\_by\_jordanification}{\M M, x}
Try to simplify a system defined by matrix $\M M$ and independent
variable $x$ by constant transformations that transform leading
expansion coefficients of $\M M$ into their Jordan forms. Return a
pair of values: the simplified matrix $\M M'$ and the transformation
$\M T$. If none of the attempted transformations reduce the
complexity of $\M M$ (as measured by $\F{matrix\_complexity}$),
return the original matrix and the identity transformation.

\functionitem{simplify\_by\_factorization}{\M M, x}
Try to simplify a system defined by matrix $\M M$ and independent
variable $x$ by a constant transformation that extracts common
factors found in $\M M$ (if any). Return a pair of values:
the simplified matrix $\M M'$ and the transformation $\M T$.

\functionitem{matrix\_complexity}{\M M}
This function is used as a measure of matrix complexity by
$\F{fuchsify}$ and simplification function. Currently it is
defined as the length of textual representation of matrix $\M M$.

\functionitem{balance}{\M P, x_1, x_2, x}
Return a \textit{balance} transformation between points $x=x_1$
and $x=x_2$ using projector matrix $\M P$.

\functionitem{transform}{\M M, x, \M T}
Transform a system defined by matrix $\M M$ and independent
variable $x$ using transformation matrix $\M T$. Return the
transformed matrix $\M M'$.

\functionitem{balance\_transform}{\M M, \M P, x_1, x_2, x}
Same as $\F{transform}(\M M, x, \F{balance}(\M P, x_1, x_2, x))$,
but implemented more efficiently: since the inverse of
$\F{balance}(\M P, x_1, x_2, x)$ is $\F{balance}(\M P, x_2, x_1, x)$,
this function can avoid a time-consuming matrix inversion
operation that $\F{transform}$ must perform.

\functionitem{singularities}{\M M, x}
Find values of $x$ around which matrix $\M M$ has a singularity in
$x$. Return a dictionary with $\{x_i: p_i\}$ entries, where $p_i$
is the Poincare rank of $\M M$ at $x=x_i$. The set of singular points
can include $+Infinity$, if $\M M$ has a singularity at $x\to\infty$.

\functionitem{matrix\_c0}{\M M, x, x_0, p}
Return the 0-th coefficient of the series expansion of matrix
$\M M$ around $x=x_0$, assuming Poincare rank of $\M M$ at that
point is $p$. If $x_0$ is $+Infinity$, return the coefficient
at the highest power of $x$.

\functionitem{matrix\_c1}{\M M, x, x_0, p}
Return the 1-th coefficient of the series expansion of matrix
$\M M$ around $x=x_0$, assuming Poincare rank of $\M M$ at that
point is $p$. If $x_0$ is $+Infinity$, return the coefficient
at the second-to-highest power of $x$.

\functionitem{matrix\_residue}{\M M, x, x_0}
Return matrix residue of matrix $\M M$ at $x=x_0$, assuming that
Poincare rank of $\M M$ at $x=x_0$ is $0$. Return matrix residue
at infinity if $x=+Infinity$.

\functionitem{export\_matrix}{file, \M M}
Write matrix $\M M$ to a file-like object $file$ using MatrixMarket
array format. For description of this format see \cite{BPR96}.

\functionitem{export\_matrix\_to\_file}{filename, \M M}
Write matrix $\M M$ to a file $filename$ using MatrixMarket
array format.

\functionitem{import\_matrix}{file}
Read a symbolic matrix from a file-like object $file$, assuming
it is formatted using MatrixMarket array format.

\functionitem{import\_matrix\_from\_file}{filename}
Read a symbolic matrix from a file $filename$, assuming it is
formatted using MatrixMarket array format.

\end{description}

\section{Summary}
\label{sec:4}

{\it Prospects: non-rational transformations (e.g. sunrise graphs). }

\section*{Acknowledgment}

\bibliography{fuchsia}

\end{document}
